%
% Halaman Abstrak
%
% @author  Andreas Febrian
% @version 1.00
%

\chapter*{Abstrak}

\vspace*{0.2cm}
{
	\setlength{\parindent}{0pt}
	
	\begin{tabular}{@{}l l p{10cm}}
		Nama&: & \penulis \\
		Program Studi&: & \program \\
		Judul&: & \judul \\
		Pembimbing&: & \pembimbing \\
	\end{tabular}

	\bigskip
	\bigskip

	Tesis ini membahas pengembangan \textit{framework Command and Control} (C2) berbasis \textit{Powershell} dengan integrasi teknik \textit{keystroke injection} yang menggunakan konsep \textit{Rubber Ducky} untuk simulasi serangan siber. Framework ini dirancang untuk mengirimkan perintah otomatis dari server C2 ke perangkat target, sehingga dapat mensimulasikan serangan berbasis injeksi perintah yang realistis dalam pengujian keamanan. Penelitian ini menggunakan pendekatan eksperimental dengan desain deskriptif untuk mengevaluasi efektivitas framework ini dalam menghindari deteksi oleh berbagai sistem keamanan.

	\bigskip

	Kata kunci:\\
	\textit{Command and Control}, \textit{PowerShell}, \textit{Keystroke Injection}, \textit{Rubber Ducky}, Simulasi Serangan Siber, Keamanan Siber
}

\newpage