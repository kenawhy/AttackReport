%-----------------------------------------------------------------------------%
\chapter{\babDua}
%-----------------------------------------------------------------------------%
\todo{tambahkan kata-kata pengantar bab 2 disini}

%-----------------------------------------------------------------------------%
\section{Command and Control (C2) Framework}
%-----------------------------------------------------------------------------%
Command and Control (C2) adalah komponen utama dalam ekosistem serangan siber yang memungkinkan peretas mengendalikan perangkat yang telah disusupi. Framework C2 dirancang untuk mengelola komunikasi antara perangkat target (client) dan server pengendali (C2 server). Dalam serangan siber, framework ini digunakan untuk mengirimkan perintah, menerima data, atau bahkan menjalankan skrip secara jarak jauh.


Framework C2 berbasis PowerShell menawarkan berbagai keuntungan, seperti fleksibilitas, kompatibilitas tinggi dengan sistem Windows, dan kemampuannya untuk menyamarkan aktivitas berbahaya. PowerShell sering dimanfaatkan karena sifatnya yang merupakan komponen bawaan Windows, sehingga meminimalkan risiko terdeteksi oleh sistem keamanan.


Framework Command and Control (C2) berperan sebagai inti dalam pengembangan penelitian ini, di mana framework ini digunakan untuk mengelola komunikasi antara server dan perangkat target. Framework berbasis PowerShell dipilih karena fleksibilitas dan kompatibilitasnya dengan sistem operasi Windows, yang merupakan sistem operasi yang sering menjadi target serangan. Dalam penelitian ini, framework C2 dikembangkan untuk mendukung pengiriman perintah dan menerima data hasil serangan melalui simulasi serangan siber. 

%-----------------------------------------------------------------------------%
\section{Keystroke Injection}
%-----------------------------------------------------------------------------%
Keystroke injection adalah metode untuk mensimulasikan input keyboard pada perangkat target secara otomatis. Teknik ini sering digunakan oleh peretas untuk menyisipkan perintah atau payload secara langsung ke perangkat korban tanpa memerlukan interaksi pengguna.


Dalam penelitian ini, Digispark digunakan sebagai perangkat untuk keystroke injection. Digispark adalah mikrokontroler berbasis ATtiny85 yang dapat diprogram melalui Arduino IDE. Dengan menggunakan library DigiKeyboard.h, Digispark mampu berfungsi sebagai keyboard USB dan mengirimkan input berupa keystroke ke perangkat target. Teknik ini sangat efektif karena perangkat target menganggap Digispark sebagai perangkat input yang sah.

Teknik keystroke injection dikaitkan langsung dengan metode pengiriman payload ke perangkat korban. Penulis menggunakan teknik ini karena kemampuannya untuk menyisipkan perintah secara langsung ke sistem target melalui input keyboard yang disimulasikan. Teknik ini relevan untuk mendukung penelitian karena memungkinkan serangan dilakukan dengan waktu singkat tanpa memerlukan instalasi perangkat lunak tambahan pada perangkat korban, sehingga meningkatkan efisiensi simulasi serangan. 

%-----------------------------------------------------------------------------%
\section{Powershell}
%-----------------------------------------------------------------------------%
PowerShell adalah shell perintah dan bahasa pemrograman yang dirancang untuk otomatisasi tugas dan konfigurasi sistem, khususnya pada sistem operasi Windows. Dalam konteks keamanan siber, PowerShell sering dimanfaatkan untuk berbagai aktivitas, seperti pengumpulan informasi, eksekusi skrip jarak jauh, dan manipulasi data.


PowerShell memiliki beberapa keunggulan dalam pengembangan framework C2, antara lain:
\begin{itemize}
    \item Kemampuan scripting yang kuat untuk membuat payload kompleks.
    \item Akses langsung ke API Windows, memungkinkan pengendalian sistem operasi.
    \item Kompatibilitas bawaan dengan sebagian besar sistem Windows, mengurangi kebutuhan untuk instalasi tambahan. 
\end{itemize}


PowerShell dipilih sebagai bahasa pemrograman utama untuk mengembangkan framework Command and Control karena sifatnya yang kuat dalam scripting dan mendukung akses langsung ke API Windows. Dalam penelitian ini, PowerShell digunakan untuk membuat payload yang disisipkan ke perangkat target melalui keystroke injection, serta untuk berkomunikasi dengan server C2 untuk mengirimkan data atau menjalankan perintah. Pemanfaatan PowerShell memungkinkan simulasi serangan yang lebih realistis dan efektif. 
%-----------------------------------------------------------------------------%
\section{Digispark}
%-----------------------------------------------------------------------------%
Digispark adalah mikrokontroler kecil berbasis ATtiny85 yang sering digunakan dalam berbagai proyek elektronik. Ukurannya yang kecil, biaya yang terjangkau, dan kompatibilitas dengan Arduino IDE menjadikan Digispark pilihan populer dalam teknik keystroke injection.


Library \verb|DigiKeyboard.h| memungkinkan Digispark untuk bertindak sebagai perangkat input USB yang mensimulasikan keyboard. Dalam penelitian ini, Digispark diprogram untuk mengirimkan perintah PowerShell ke perangkat target melalui simulasi input keyboard. Keunggulan Digispark adalah kemampuannya untuk menjalankan tugas ini dengan cepat dan tanpa perlu modifikasi besar pada perangkat target.

 
Penulis menggunakan mikrokontroler Digispark karena perangkat keras ini memiliki library \verb|DigiKeyboard.h| yang memungkinkan penulis untuk melakukan simulasi keystroke injection pada perangkat korban. Digispark dipilih karena ukurannya yang kecil, biaya yang rendah, dan kompatibilitas dengan Arduino IDE, sehingga mempermudah pengembangan perangkat untuk menyisipkan perintah PowerShell ke perangkat target. Dalam penelitian ini, Digispark menjadi komponen penting untuk mengintegrasikan perangkat keras dengan framework C2. 

%-----------------------------------------------------------------------------%
\section{Amazon AWS}
%-----------------------------------------------------------------------------%
Amazon Web Services (AWS) adalah platform cloud yang menyediakan berbagai layanan, termasuk server yang dapat digunakan sebagai pusat Command and Control (C2). Dalam penelitian ini, AWS digunakan untuk: 
\begin{itemize}
    \item Menyediakan server yang dapat menerima dan mengolah data dari perangkat target.
    \item Mendukung komunikasi aman antara perangkat target dan server melalui protokol HTTPS.
    \item Mengelola penyimpanan data hasil serangan untuk analisis lebih lanjut. 
\end{itemize}
 

AWS digunakan sebagai server Command and Control karena kemampuannya menyediakan infrastruktur yang skalabel dan aman untuk mendukung penelitian. Server AWS berfungsi sebagai pusat kendali untuk menerima data dari perangkat korban dan mengirimkan perintah selama simulasi serangan. Korelasinya dengan penelitian adalah memberikan fleksibilitas dalam pengelolaan server C2, serta mendukung pengujian berbagai skenario serangan dengan komunikasi aman melalui protokol HTTPS. 
%-----------------------------------------------------------------------------%
