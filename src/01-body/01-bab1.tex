%-----------------------------------------------------------------------------%
\chapter{\babSatu}



%-----------------------------------------------------------------------------%
\section{Latar Belakang}
%-----------------------------------------------------------------------------%

Di era digital yang semakin maju, ancaman siber menjadi salah satu tantangan terbesar dalam menjaga keamanan informasi\citep{gade2016cyberattacks}. Serangan siber tidak hanya menargetkan individu, tetapi juga perusahaan, institusi pemerintah, dan infrastruktur kritis. Berbagai teknik dan alat telah dikembangkan oleh peretas untuk mengeksploitasi kerentanan sistem, termasuk penggunaan Command and Control (C2) frameworks yang memungkinkan kendali jarak jauh terhadap perangkat yang terinfeksi.

Command and Control (C2) adalah bagian penting dari ekosistem serangan siber, di mana peretas dapat mengirimkan perintah, menerima data, atau memodifikasi perilaku perangkat yang disusupi\citep{hutchins2011killchain}. Framework berbasis PowerShell telah menjadi populer di kalangan peretas karena sifatnya yang fleksibel, mudah disembunyikan, dan banyaknya dukungan bawaan dari sistem operasi Windows\citep{powershell_c2}. Namun, kerentanan ini juga dapat dimanfaatkan untuk meningkatkan pemahaman dan pengembangan sistem keamanan melalui simulasi serangan.

Salah satu teknik pendukung yang semakin sering digunakan dalam serangan siber adalah keystroke injection\citep{usb_attack_digispark}. Teknik ini memanfaatkan perangkat kecil seperti Digispark, yang dapat diprogram untuk mensimulasikan input keyboard secara otomatis ke perangkat target. Dengan teknik ini, peretas dapat menanamkan perintah langsung ke perangkat korban tanpa memerlukan akses fisik yang lama, sehingga meningkatkan efektivitas serangan.

Berdasarkan latar belakang tersebut, penelitian ini bertujuan untuk mengembangkan framework Command and Control berbasis PowerShell dengan integrasi teknik keystroke injection menggunakan Digispark. Framework ini akan digunakan untuk simulasi serangan siber, yang diharapkan dapat memberikan kontribusi bagi peningkatan kesadaran, deteksi, dan mitigasi ancaman siber di masa depan\citep{nist_framework}.


%-----------------------------------------------------------------------------%
\section{Rumusan Masalah}
%-----------------------------------------------------------------------------%
Pada bagian ini akan dijelaskan mengenai definisi permasalahan 
yang \saya~hadapi:
\begin{enumerate}
    \item Bagaimana mengembangkan framework Command and Control (C2) berbasis PowerShell yang dapat digunakan untuk simulasi serangan siber?
    \item Bagaimana mengintegrasikan teknik keystroke injection menggunakan Digispark ke dalam framework C2?
    \item Seberapa efektif framework yang dikembangkan dalam mensimulasikan serangan siber untuk menguji ketahanan sistem keamanan?
\end{enumerate}


%-----------------------------------------------------------------------------%
\section{Tujuan}
%-----------------------------------------------------------------------------%
Adapun tujuan yang ingin dicapai dalam penelitian ini adalah:

\begin{enumerate}
    \item Mengembangkan framework Command and Control (C2) berbasis PowerShell yang dapat digunakan untuk simulasi serangan siber.
    \item Mengintegrasikan teknik keystroke injection menggunakan Digispark ke dalam framework C2 untuk meningkatkan efisiensi simulasi serangan.
\end{enumerate}

%-----------------------------------------------------------------------------%
\section{Metodologi Penelitian}
%-----------------------------------------------------------------------------%
Penelitian ini dilakukan dengan menggunakan metode pengembangan dan pengujian yang terdiri dari tahapan berikut:

 \begin{enumerate}
     \item \textbf{Studi Literatur}

Mengumpulkan dan mempelajari referensi terkait Command and Control (C2) frameworks, PowerShell, teknik keystroke injection, dan perangkat Digispark.

\item \textbf{Desain dan Pengembangan Framework}

Merancang framework C2 berbasis PowerShell dengan mengintegrasikan teknik keystroke injection menggunakan Digispark.

Implementasi modul-modul utama untuk pengiriman perintah, penerimaan data, dan eksekusi serangan jarak jauh.

 
\item \textbf{Pengaturan Lingkungan Uji}

Menyiapkan lingkungan uji berupa server C2 dan perangkat target untuk mensimulasikan serangan.

 Mengonfigurasi jaringan untuk mendukung komunikasi antara server dan perangkat target.

 
\item \textbf{Simulasi dan Pengujian}

Melakukan simulasi serangan menggunakan framework yang telah dikembangkan.

 Menguji efektivitas framework dalam berbagai skenario serangan siber.

 Mengukur keberhasilan komunikasi C2 dan akurasi teknik keystroke injection.

 
\item \textbf{Evaluasi dan Analisis}

Mengevaluasi hasil pengujian berdasarkan parameter yang telah ditentukan, seperti tingkat keberhasilan serangan dan waktu eksekusi.

\item \textbf{Dokumentasi}

Mendokumentasikan seluruh tahapan penelitian, hasil pengujian, dan analisis untuk mendukung laporan akhir skripsi. 

 \end{enumerate}


 
%-----------------------------------------------------------------------------%
\section{Sistematika Penulisan}
%-----------------------------------------------------------------------------%
Sistematika penulisan laporan adalah sebagai berikut:
\begin{itemize}
	\item Bab 1 \babSatu \\
    Bab ini berisi latar belakang penelitian, rumusan masalah, tujuan penelitian, batasan masalah, dan sistematika penulisan yang menjelaskan struktur dari laporan penelitian.
    
	\item Bab 2 \babDua \\
    Bab ini menguraikan konsep dan teori yang relevan sebagai landasan penelitian, seperti Command and Control (C2) frameworks, PowerShell, keystroke injection, dan perangkat Digispark.

	\item Bab 3 \babTiga \\
    Bab ini menjelaskan proses desain dan implementasi framework, termasuk integrasi teknik keystroke injection menggunakan Digispark.
    
	\item Bab 4 \babEmpat \\
    Bab ini memaparkan hasil pengujian framework yang dikembangkan, serta analisis efektivitas dan evaluasi berdasarkan parameter yang telah ditentukan.
    
	\item Bab 5 \babLima \\
    Bab ini berisi kesimpulan dari penelitian yang dilakukan serta saran untuk pengembangan lebih lanjut dan aplikasi praktis di masa depan.

\end{itemize}


